The answer is $\frac{2n}{3}$ if $3\mid n$ and $2n$ if else.

First, we prove that these are lower bounds. For each row or column, count the number of dominoes that lie on it. Because the configuration is balanced, this is $k$ for each row or column, so the total sum is $2nk$. Meanwhile, each domino is counted in this sum exactly $3$ times (once for the row/column which is is completely contained within, once each for the rows/columns which it only hits once), so the number of dominoes is $\frac{2nk}{3}$. When $3\mid n$, we have $k\geq1$ so $\frac{2nk}{3}\geq\frac{2n}{3}$. When $3\nmid n$, $k\geq3$ for divisibility, so $\frac{2nk}{3}\geq2n$.

Now, we construct these bounds. For $3\mid n$, do the following:
\begin{center}
\begin{asy}
size(6cm);
real d = 0.1;
int n = 6;
path rect(int a, int b, int x, int y)
{
return (a-1+d,b-1+d)--(x-d,b-1+d)--(x-d,y-d)--(a-1+d,y-d)--cycle;
}
for (int i = 0; i <= n; ++i)
{
draw((i,0)--(i,n));
draw((0,i)--(n,i));
}
fill(rect(1,4,1,5),red);
fill(rect(2,6,3,6),red);
fill(rect(4,1,4,2),red);
fill(rect(5,3,6,3),red);
\end{asy}
\end{center}
(continued in a block-diagonal repetition of the formation $\frac{n}{3}$ times).

So it suffices to show that every $n\geq3$ not divisible by $3$ has a balanced configuration with $2n$ dominoes. We go further and show that every $n\geq4$ except $6$ has a balanced configuration with $2n$ dominoes (equivalently with $k=3$). First, let us construct it for $n=4,5,7$:

\begin{center}
\begin{asy}
size(4cm);
real d = 0.1;
int n = 4;
path rect(int a, int b, int x, int y)
{
return (a-1+d,b-1+d)--(x-d,b-1+d)--(x-d,y-d)--(a-1+d,y-d)--cycle;
}
for (int i = 0; i <= n; ++i)
{
draw((i,0)--(i,n));
draw((0,i)--(n,i));
}
fill(rect(1,1,1,2),red);
fill(rect(2,1,2,2),red);
fill(rect(1,3,2,3),red);
fill(rect(1,4,2,4),red);
fill(rect(3,1,4,1),red);
fill(rect(3,2,4,2),red);
fill(rect(3,3,3,4),red);
fill(rect(4,3,4,4),red);
\end{asy}
\end{center}
\begin{center}
\begin{asy}
size(5cm);
real d = 0.1;
int n = 5;
path rect(int a, int b, int x, int y)
{
return (a-1+d,b-1+d)--(x-d,b-1+d)--(x-d,y-d)--(a-1+d,y-d)--cycle;
}
for (int i = 0; i <= n; ++i)
{
draw((i,0)--(i,n));
draw((0,i)--(n,i));
}
fill(rect(1,1,1,2),red);
fill(rect(1,3,1,4),red);
fill(rect(1,5,2,5),red);
fill(rect(2,2,3,2),red);
fill(rect(2,3,3,3),red);
fill(rect(3,1,4,1),red);
fill(rect(4,3,5,3),red);
fill(rect(4,4,4,5),red);
fill(rect(5,4,5,5),red);
fill(rect(5,1,5,2),red);
\end{asy}
\end{center}
\begin{center}
\begin{asy}
size(7cm);
real d = 0.1;
int n = 7;
path rect(int a, int b, int x, int y)
{
return (a-1+d,b-1+d)--(x-d,b-1+d)--(x-d,y-d)--(a-1+d,y-d)--cycle;
}
for (int i = 0; i <= n; ++i)
{
draw((i,0)--(i,n));
draw((0,i)--(n,i));
}
fill(rect(1,2,1,3),red);
fill(rect(1,5,1,6),red);
fill(rect(1,7,2,7),red);
fill(rect(2,1,2,2),red);
fill(rect(2,4,2,5),red);
fill(rect(3,1,4,1),red);
fill(rect(3,4,4,4),red);
fill(rect(3,6,3,7),red);
fill(rect(4,5,5,5),red);
fill(rect(5,2,6,2),red);
fill(rect(5,3,5,4),red);
fill(rect(6,1,7,1),red);
fill(rect(6,3,7,3),red);
fill(rect(7,6,7,7),red);
\end{asy}
\end{center}

Now, I claim that the set of $n$ that have a balanced configuration with $k=3$ is closed under addition. Indeed, if $n_1$ and $n_2$ have this property, then just construct $n_1+n_2$ by appending the construction for $n_2$ to that of $n_1$ in a block-diagonal fashion. As an example, here is the construction for $11=4+7$:

\begin{center}
\begin{asy}
size(11cm);
real d = 0.1;
int n = 11;
path rect(int a, int b, int x, int y)
{
return (a-1+d,b-1+d)--(x-d,b-1+d)--(x-d,y-d)--(a-1+d,y-d)--cycle;
}
for (int i = 0; i <= n; ++i)
{
draw((i,0)--(i,n));
draw((0,i)--(n,i));
}
fill(rect(1,8,1,9),red);
fill(rect(2,8,2,9),red);
fill(rect(1,10,2,10),red);
fill(rect(1,11,2,11),red);
fill(rect(3,8,4,8),red);
fill(rect(3,9,4,9),red);
fill(rect(3,10,3,11),red);
fill(rect(4,10,4,11),red);
fill(rect(5,2,5,3),red);
fill(rect(5,5,5,6),red);
fill(rect(5,7,6,7),red);
fill(rect(6,1,6,2),red);
fill(rect(6,4,6,5),red);
fill(rect(7,1,8,1),red);
fill(rect(7,4,8,4),red);
fill(rect(7,6,7,7),red);
fill(rect(8,5,9,5),red);
fill(rect(9,2,10,2),red);
fill(rect(9,3,9,4),red);
fill(rect(10,1,11,1),red);
fill(rect(10,3,11,3),red);
fill(rect(11,6,11,7),red);
\end{asy}
\end{center}

This works because each of the first $n_1$ rows and columns still only have $3$ dominoes, and same with each of the last $n_2$ rows and columns.

Since $4$ and $5$ have this property, by the [b]Chicken McNugget Theorem[/b], all integers larger than $4\cdot5-4-5=11$ have this property. So it suffices to check $4,5,7,8,9,10,11$ have this property. We have already constructed $4,5,7$ and have $8=4+4,9=4+5,10=5+5,11=4+7$, so our claim is true.

So a construction for the lower bound provided exists and thus the minimum is indeed as claimed at the beginning.