Note that \[m=\frac{\frac{n\left(n+1\right)\left(2n+1\right)}{6}}{\frac{n\left(n+1\right)}{2}}=\frac{2n+1}{3},\] so $m$ is an integer if and only if $n$ leaves a remainder of $1$ when divided by $3$. Furthermore, $\frac{2n+1}{3}<2017$, so $n<3025$. Thus, we must count the number of $n\in\left(2017,3025\right)$ such that $n$ leaves a remainder of $1$ when divided by $3$. Note that $2020=3\cdot673+1$ and $3022=3\cdot1007+1$, so there are $1007-673+1=\boxed{335}$ possible $n$.