Let $P_n$ be the probability that the middle cup contains the ball after $n$ flips. Then $P_0=1$ and $P_1=0$ by observation. Furthermore, note that if the middle cup contains the ball after $n$ flips, then it cannot after $n+1$ flips, and if the middle cup does not contain the ball after $n$ flips, then it does with probability $\frac{1}{2}$ after $n+1$ flips. Thus, we deduce that \[P_{n+1}=\frac{1}{2}\left(1-P_n\right).\] Let $Q_n=P_n-\frac{1}{3}$. Substituting this in gives us that $Q_0=\frac{2}{3}$, $Q_1=-\frac{1}{3}$, and $Q_{n+1}=-\frac{1}{2}Q_n$ for all $n$. Then it is clear that $Q$ is a geometric sequence represented by $\frac{2}{3}\left(-\frac{1}{2}\right)^n$, so we see that $P_n=\frac{1}{3}+\frac{2}{3}\left(-\frac{1}{2}\right)^n$. Plugging in $n=2017$ gives us that $P_n=\boxed{\frac{2^{2016}-1}{3\cdot2^{2016}}}$.