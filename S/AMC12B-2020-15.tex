The answer is $\boxed{F_n+2F_{n-1}+2(n\%2)}$, where $n\%2$ is $1$ for odd $n$ and $0$ for even $n$. Here, $\{F_k\}$ is the Fibonacci sequence satisfying $F_1=F_2=1$ and $F_{k+1}=F_k+F_{k-1}$.

Label the people $P_1,\ldots,P_{2n}$ with indices taken mod $2n$. Casework on who $P_0$ is paired with.
\begin{itemize}
	\item \underline{$P_0$ paired with $P_1$}
	
	If $P_n$ is paired with $P_{n+1}$, then we are left with two symmetrically opposite sections of $n-2$ people.
	\begin{center}
		\begin{asy}
			int n = 11;
			int[][] edges = {
				{0, 1},
				{n, n + 1},
			};
			int[][] groups = {
				{2, n - 1},
				{2 + n, n - 1 + n},
			};
			int[] labelIndex = {
				0,
				1,
				n,
				n + 1,
			};
			string[] labelName = {
				"0",
				"1",
				"n",
				"n+1",
			};
			real s = 1.17;
			
			pen edgeColor = blue;
			size(5cm);
			pair[] P;
			for (int i = 0; i < 2 * n; ++i) {
				P.push(dir(90 + 180 * i / n));
				dot(P[i]);
			}
			for (int[] edge : edges) draw(P[edge[0]]--P[edge[1]], edgeColor);
			
			pen groupColor = purple;
			real r = Sin(90 / n);
			pair O = (0,0);
			for (int[] group : groups) {
				draw(arc(P[group[0]], r, -90 + 180 * group[0] / n, 90 + 180 * group[0] / n, CCW), groupColor);
				draw(arc(P[group[1]], r, -90 + 180 * group[1] / n, 90 + 180 * group[1] / n, CW), groupColor);
				draw(arc(O, 1 + r, 90 + 180 * group[0] / n, 90 + 180 * group[1] / n), groupColor);
				draw(arc(O, 1 - r, 90 + 180 * group[0] / n, 90 + 180 * group[1] / n), groupColor);
			}
			
			defaultpen(fontsize(10pt));
			for (int i = 0; i < labelIndex.length; ++i) label ("$P_{" + labelName[i] + "}$", s * P[labelIndex[i]]);
		\end{asy}
	\end{center}
	The key claim is that the figure is symmetric under $180^\circ$ rotation. It suffices to check that neighbor pairings are symmetric. First pair up all opposite pairings. This leaves some gaps of even length.
	\begin{center}
		\begin{asy}
			int n = 11;
			int[][] edges = {
				{0, 1},
				{n, n + 1},
				{2, 2 + n},
				{3, 3 + n},
				{n - 3, n - 3 + n},
			};
			int[][] groups = {
				{4, n - 4},
				{n - 2, n - 1},
				{4 + n, n - 4 + n},
				{n - 2 + n, n - 1 + n},
			};
			int[] labelIndex = {
				0,
				1,
				n,
				n + 1,
			};
			string[] labelName = {
				"0",
				"1",
				"n",
				"n+1",
			};
			real s = 1.2;
			
			pen edgeColor = blue;
			size(5cm);
			pair[] P;
			for (int i = 0; i < 2 * n; ++i) {
				P.push(dir(90 + 180 * i / n));
				dot(P[i]);
			}
			for (int[] edge : edges) {
				draw(P[edge[0]]--P[edge[1]], edgeColor);
			}
			
			pen groupColor = green;
			real r = Sin(90 / n);
			pair O = (0,0);
			for (int[] group : groups) {
				draw(arc(P[group[0]], r, -90 + 180 * group[0] / n, 90 + 180 * group[0] / n, CCW), groupColor);
				draw(arc(P[group[1]], r, -90 + 180 * group[1] / n, 90 + 180 * group[1] / n, CW), groupColor);
				draw(arc(O, 1 + r, 90 + 180 * group[0] / n, 90 + 180 * group[1] / n), groupColor);
				draw(arc(O, 1 - r, 90 + 180 * group[0] / n, 90 + 180 * group[1] / n), groupColor);
			}
			
			defaultpen(fontsize(10pt));
			for (int i = 0; i < labelIndex.length; ++i) label ("$P_{" + labelName[i] + "}$", s * P[labelIndex[i]]);
		\end{asy}
	\end{center}
	These gaps have only one way to be filled in, so they are indeed symmetric. Thus it suffices to fill in one section of $n-2$ with blocks of $2$ or $1$, corresponding to neighbor pairings or opposite pairings. It is well-known that this can be done in $F_{n-1}$ ways.
	
	Otherwise, $P_n$ is paired with $P_{n-1}$. We inductively prove that $(P_{-2k+1},P_{-2k})$ and $(P_{n-2k},P_{n-2k-1})$ are pairings by induction on $k$. The base case of $k=0$ is assumed. Check that given the pairings for $k-1$, the pairings for $k$ are forced:
	\begin{center}
		\begin{asy}
			int n = 11;
			int k = 1;
			int[][] edges = {
				{- 2 * (k - 1) + 1, - 2 * (k - 1)},
				{n - 2 * (k - 1), n - 2 * (k - 1) - 1},
			};
			int[][] fakeEdges = {
				{2 * n - 2 * k + 1, - 2 * (k - 1)},
				{2 * n - 2 * k + 1, n - 2 * k + 1},
				{n - 2 * k, 2 * n - 2 * k},
				{n - 2 * k, n - 2 * k + 1},
			};
			int[][] newEdges = {
				{2 * n - 2 * k + 1, 2 * n - 2 * k},
				{n - 2 * k, n - 2 * k - 1},
			};
			
			pen edgeColor = blue;
			size(5cm);
			pair[] P;
			for (int i = 0; i < 2 * n; ++i) {
				P.push(dir(90 + 180 * i / n));
				dot(P[i]);
			}
			for (int[] edge : edges) {
				draw(P[edge[0]]--P[edge[1]], edgeColor);
			}
			
			pen fakeEdgeColor = red;
			for (int[] fakeEdge : fakeEdges) {
				draw(P[fakeEdge[0]]--P[fakeEdge[1]], fakeEdgeColor+dashed);
			}
			
			pen newEdgeColor = rgb(0,191,255);
			for (int[] newEdge : newEdges) {
				draw(P[newEdge[0]]--P[newEdge[1]], newEdgeColor);
			}
		\end{asy}
	\end{center}
	So only neighbors are paired. Then $0$ and $n-1$ must have the same parity, so $n$ is odd, with only one case.
	
	Combining these, there are $F_{n-1}+(n\%2)$ possible pairings in this case.
	\item \underline{$P_0$ paired with $P_{-1}$}
	
	Symmetric with the previous case.
	\item \underline{$P_0$ paired with $P_n$}
	
	We are left with two symmetrically opposite sections of $n-1$ people.
	\begin{center}
		\begin{asy}
			int n = 11;
			int[][] edges = {
				{0, n},
			};
			int[][] groups = {
				{1, n - 1},
				{1 + n, n - 1 + n},
			};
			int[] labelIndex = {
				0,
				n,
			};
			string[] labelName = {
				"0",
				"n",
			};
			real s = 1.17;
			
			pen edgeColor = blue;
			size(5cm);
			pair[] P;
			for (int i = 0; i < 2 * n; ++i) {
				P.push(dir(90 + 180 * i / n));
				dot(P[i]);
			}
			for (int[] edge : edges) draw(P[edge[0]]--P[edge[1]], edgeColor);
			
			pen groupColor = purple;
			real r = Sin(90 / n);
			pair O = (0,0);
			for (int[] group : groups) {
				draw(arc(P[group[0]], r, -90 + 180 * group[0] / n, 90 + 180 * group[0] / n, CCW), groupColor);
				draw(arc(P[group[1]], r, -90 + 180 * group[1] / n, 90 + 180 * group[1] / n, CW), groupColor);
				draw(arc(O, 1 + r, 90 + 180 * group[0] / n, 90 + 180 * group[1] / n), groupColor);
				draw(arc(O, 1 - r, 90 + 180 * group[0] / n, 90 + 180 * group[1] / n), groupColor);
			}
			
			defaultpen(fontsize(10pt));
			for (int i = 0; i < labelIndex.length; ++i) label ("$P_{" + labelName[i] + "}$", s * P[labelIndex[i]]);
		\end{asy}
	\end{center}
	As before, there are $F_{2k}$ ways for this to be filled out.
\end{itemize}
Combining the three cases, we get the desired answer.