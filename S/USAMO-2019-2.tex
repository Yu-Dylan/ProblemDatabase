If $M$ is the midpoint of $CD$, let $Q$ be where $EM$ and $AB$ meet. The goal is to show that $P=Q$. Note that since $P$ is unique (try moving $P$ along segment $AB$), it suffices to show $\angle{AQD}=\angle{BQC}$.

Observe that since $M,E,Q$ collinear, $EQ$ and $EM$ are corresponding isogonal lines in similar triangles $ECD$ and $EAB$. So $EQ$ is a symmedian and thus $\frac{AQ}{QB}=\frac{AE^2}{EB^2}$. But by the Law of Sines, \[\frac{AE}{EB}=\frac{\sin\angle{ABE}}{\sin\angle{EAB}}=\frac{\sin\angle{ABD}}{\sin\angle{CAB}}=\frac{AD}{BC}\] so $\frac{AQ}{QB}=\frac{AD^2}{BC^2}$. Since $AQ+QB=AB$, it follows that $AQ=\frac{AD^2}{AB}$. Then $AD$ is tangent to $(BQD)$ so $\angle{QDA}=\angle{ABD}$. Then \[\angle{AQD}=\pi-\angle{QDA}-\angle{DAQ}=\pi-\angle{ABD}-\angle{DAB}=\angle{BDA}.\] Similarly, $\angle{BQC}=\angle{ACB}$. But $\angle{BDA}=\angle{ACB}$ by cyclic quad, so $\angle{AQD}=\angle{BQC}$ as desired.