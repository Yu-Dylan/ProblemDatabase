Let $M_C$ be the second intersection of $\Gamma$ and the angle bisector of $\angle{BCA}$. Invert about $M_C$ with radius $M_CA$. Then $\Gamma$ and line $AB$ get swapped, while the center of $\Omega$ and the center of its image must lie on a line through $M_C$. But there is only one circle centered on this line in this position that is tangent to $AB$ and internally tangent to $\Gamma$, so $\Omega$ is fixed under this inversion. So the power of $M_C$ with respect to $\Omega$ is $M_CA$. But it is also $M_CP\cdot M_CQ$, so $\triangle{M_CAP}\sim\triangle{M_CQA}$. So
\begin{align*}
	\measuredangle{PAC}&=\measuredangle{BAC}-\measuredangle{BAP}\\
	&=\measuredangle{BAC}-\measuredangle{BAM_C}-\measuredangle{M_CAP}\\
	&=\measuredangle{BAC}-\measuredangle{BCM_C}+\measuredangle{APM_C}+\measuredangle{PM_CA}\\
	&=\measuredangle{BAC}-\measuredangle{BCM_C}+\measuredangle{APM_C}+\measuredangle{CM_CA}\\
	&=\measuredangle{BAC}-\measuredangle{BCM_C}+\measuredangle{APM_C}+\measuredangle{CBA}\\
	&=\measuredangle{BCA}+\measuredangle{M_CCB}+\measuredangle{APM_C}\\
	&=\measuredangle{M_CCA}+\measuredangle{APM_C}\\
	&=\measuredangle{BCM_C}+\measuredangle{APM_C}\\
	&=\measuredangle{BAM_C}+\measuredangle{M_CAQ}\\
	&=\measuredangle{BAQ},
\end{align*}
so $AP$ and $AQ$ are isogonal. But $CP$ and $CQ$ are isogonal, so $P$ and $Q$ are isogonal conjugates. Then $BP$ and $BQ$ are isogonal, so $\angle{ABP}=\angle{QBC}$.