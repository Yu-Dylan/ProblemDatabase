We will apply Ostrowski's Theorem. This theorem states the following: Let $P$ be a polynomial in $x_1,\ldots,x_n$. For $i=1,\ldots,n$, let $e_i$ be the largest exponent of $x_i$ to appear in $P$. Suppose there exist $a_1,\ldots,a_n\in\mathbb{Z}$ such that $P(x_1,\ldots,x_n)\in\mathbb{Z}$ whenever $x_i\in\{a_i,a_i+1,\ldots,a_i+e_i\}$ for $i=1,\ldots,n$. Then $P(x_1,\ldots,x_n)\in\mathbb{Z}$ whenever $x_1,\ldots,x_n\in\mathbb{Z}$.

Let \[P(x_1,\ldots,x_n)=\prod_{1\leq i<j\leq n}\frac{x_i-x_j}{i-j}\] as a polynomial, then we have that $e_i=n-1$ for $i=1,\ldots,n$. Now, if we choose $a_i=1$ for $i=1,\ldots,n$, we must show that $P(x_1,\ldots,x_n)\in\mathbb{Z}$ whenever $x_i\in\{1,\ldots,n\}$ for $i=1,\ldots,n$.
\begin{itemize}
	\item If there are indices $1\leq i<j\leq n$ such that $x_i=x_j$, then $P(x_1,\ldots,x_n)=0$ by definition. 
	\item Otherwise, $(x_1,\ldots,x_n)$ is a permutation of $(1,\ldots,n)$. Let $\pi:\{1,\ldots,n\}\to\{1,\ldots,n\}$ be the permutation with $\pi(i)=x_i$ for $i=1,\ldots,n$; define $\pi^{-1}$ as its inverse. Then \[|P(x_1,\ldots,x_n)|=\prod_{1\leq i<j\leq n}\frac{|\pi(i)-\pi(j)|}{|i-j|}.\] Now we exhibit a bijection between ordered pairs $(i,j)$ with $i<j$ and $(a,b)$ with $a<b$ such that $|i-j|=|\pi(a)-\pi(b)|$. This bijection is just to take $i=\min\{\pi(a),\pi(b)\}$ and $j=\max\{\pi(a),\pi(b)\}$, equivalently $a=\min\{\pi^{-1}(i),\pi^{-1}(j)\}$ and $b=\max\{\pi^{-1}(i),\pi^{-1}(j)\}$. Because of this bijection, the product of the numerators equals the product of the denominators and thus $|P(x_1,\ldots,x_n)|=1$.
\end{itemize}
In all cases, $P(x_1,\ldots,x_n)\in\mathbb{Z}$ whenever $x_i\in\{1,\ldots,n\}$ for $i=1,\ldots,n$. So by Ostrowski's Theorem, $P(x_1,\ldots,x_n)\in\mathbb{Z}$ whenever $x_1,\ldots,x_n\in\mathbb{Z}$. It follows that \[\prod_{1\leq i<j\leq n}\frac{a_i-a_j}{i-j}\] is an integer for any integers $a_1,\ldots,a_n$.