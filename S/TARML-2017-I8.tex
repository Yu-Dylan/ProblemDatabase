Note that $D$ is the $A$-excenter of $\triangle{ABC}$. Let $F$ and $G$ be the tangency points of the $A$-excircle of $\triangle{ABC}$ onto $AC$ and $AB$, respectively. Let $R$ be the common point of $BC$, $PQ$, and the line through $D$ perpendicular to $AE$. Let $R'$ be the point where $AE$ and $DR$ intersect (note that $AR'\perp DR$), and let $A'$ be the point where $AD$ and $FG$ intersect (note that $AR'\perp FG$). I claim that $AA'R'R$ is cyclic. Considering right triangle $\triangle{DER}$, we get that $DR'=\frac{DE^2}{DR}$. Considering right triangle $\triangle{DFA}$, we get that $DA'=\frac{DF^2}{DA}$. Thus, \[DR'\cdot DR=DE^2=DF^2=DA'\cdot DA,\] so $AA'R'R$ is cyclic. But then because $\angle{AR'R}=\frac{\pi}{2}$, we have that $\angle{AA'R}=\frac{\pi}{2}$, so $R$ lies on the line perpendicular to $AA'$ through $A'$. But this line is precisely $FG$, so $F$, $G$, and $R$ are collinear. By Menelaus' Theorem on $\triangle{ABC}$ with line $RFG$, \[\frac{CF}{FA}\cdot\frac{AG}{GB}=\frac{CR}{RB}.\] By Menelaus' Theorem on $\triangle{ABC}$ with line $RQP$, \[\frac{CQ}{QA}\cdot\frac{AP}{PB}=\frac{CR}{RB},\] so \[\frac{CF}{FA}\cdot\frac{AG}{GB}=\frac{CQ}{QA}\cdot\frac{AP}{PB}.\] Let $AP=x$ and $AQ=y$. Note that $CF=CE=7$, $FA=AG$, and $GB=10$. Thus, \[\frac{7}{10}=\frac{23-y}{y}\cdot\frac{x}{20-x}.\] Rearranging, this is \[3xy-230x+140y=0.\] Taking this modulo $230$, we get that $3y\left(x-30\right)$ is divisible by $230$. In particular, $y\left(x-30\right)$ is divisible by $23$. Since $y<23$, we have that $x-30$ is divisible by $23$. Since $0<x<20$, $x=7$. Plugging this in, $y=10$. Then the area of $\triangle{APQ}$ is \[\frac{7\cdot10}{20\cdot23}\sqrt{30\cdot10\cdot13\cdot7}=\boxed{\frac{35\sqrt{273}}{23}}.\]

Note: The work done to get $F$, $G$, $R$ collinear can also just be done with La Hire's Theorem: note that $A$ is on the polar of $R$, so $R$ is on the polar of $A$, which is $FG$. In fact, the work done in the solution above is just the proof of La Hire's Theorem.