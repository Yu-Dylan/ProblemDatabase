Write the equation as \[\frac{mn}{3}=2\left(m+n\right)-2.\] Rearrange this to get that \[mn-6m-6n+6=0.\] Note that this is equivalent to \[\left(m-6\right)\left(n-6\right)=30.\] Then $\left(m-6,n-6\right)$ is one of $\left(1,30\right)$, $\left(2,15\right)$, $\left(3,10\right)$, $\left(5,6\right)$, or their permutations. Since congruent rectangles are non-distinct, the permutations do not matter. Note that each of these pairs gives a valid $\left(m,n\right)$, so there are $\boxed{4}$ possible rectangles.