Let $g_p=\displaystyle\sum_{n=0}^{p-1}\left(\frac{n}{p}\right)\zeta^n$.

The first step is to prove that $g_p^2=(-1)^{\frac{p-1}{2}}p$. To do this, observe that
\begin{align*}
	g_p\overline{g_p} &= \sum_{n=0}^{p-1}\sum_{m=0}^{p-1}\left(\frac{nm}{p}\right)\zeta^{n-m} \\
	&= \sum_{d=0}^{p-1}\zeta^d\sum_{n=0}^{p-1}\left(\frac{n(n-d)}{p}\right) \\
	&= p-1+\sum_{d=1}^{p-1}\zeta^d\sum_{n=1}^{p-1}\left(\frac{n(n-d)}{p}\right) \\
	&= p-1+\sum_{d=1}^{p-1}\zeta^d\sum_{n=1}^{p-1}\left(\frac{1-\frac{d}{n}}{p}\right) \\
	&= p-1+\sum_{d=1}^{p-1}\zeta^d\sum_{\substack{0\leq e\leq p-1 \\ e\neq1}}\left(\frac{e}{p}\right) \\
	&= p-1+\sum_{d=1}^{p-1}(-\zeta^d) \\
	&= p.
\end{align*}
But
\[
	\overline{g_p}=\sum_{m=0}^{p-1}\left(\frac{m}{p}\right)\zeta^{-m}=\sum_{m=0}^{p-1}\left(\frac{-m}{p}\right)\zeta^m=(-1)^{\frac{p-1}{2}}g_p
\]
so $g_p^2=(-1)^{\frac{p-1}{2}}p$ as desired.

Now, define polynomials
\begin{align*}
	G(X) &= \sum_{n=0}^{p-1}\left(\frac{n}{p}\right)X^n \\
	H(X) &= \prod_{k=1}^{\frac{p-1}{2}}\left(X^{-k/2}-X^{k/2}\right)
\end{align*}
where the exponents in $h$ are taken mod $p$.

We know that $G(\zeta)^2=p^*$. Observe that
\begin{align*}
	H(\zeta)^2 &= \prod_{k=1}^{\frac{p-1}{2}}\left(\zeta^{-k/2}-\zeta^{k/2}\right)^2 = \prod_{k=1}^{\frac{p-1}{2}}\left(\zeta^{-k}-1\right)\left(1-\zeta^k\right) \\
	&= (-1)^{\frac{p-1}{2}}\prod_{k=1}^{p-1}\left(1-\zeta^k\right)=(-1)^{\frac{p-1}{2}}\Phi_p(1)=(-1)^{\frac{p-1}{2}}p
\end{align*}
so $G(\zeta)^2=H(\zeta)^2$. It follows that $G(\zeta)=\eps H(\zeta)$ for some $\eps\in\{\pm1\}$ and thus $\zeta$ is a root of the polynomial $G-\eps H$. Thus $\Phi_p$ divides $G-\eps H$.

Now, we work in $\mathbb{F}_p$. First note that
\begin{align*}
	G(1+Y) &= \sum_{n=0}^{p-1}\left(\frac{n}{p}\right)(1+Y)^n \\
	&= \sum_{n=0}^{p-1}\sum_{m=0}^n\left(\frac{n}{p}\right)\binom{n}{m}Y^m \\
	&= \sum_{m=0}^{p-1}\left(\sum_{n=m}^{p-1}\binom{n}{m}\left(\frac{n}{p}\right)\right)Y^m.
\end{align*}
Suppose that $m<\frac{p-1}{2}$ and consider the inside sum. Let $\binom{X}{m}=\frac{1}{m!}\displaystyle\sum_{j=0}^ma_{m,j}X^j$ be the binomial coefficient polynomial. Then
\[
	\sum_{n=m}^{p-1}\binom{n}{m}\left(\frac{n}{p}\right)=\sum_{n=0}^{p-1}\binom{n}{m}\left(\frac{n}{p}\right)=\sum_{n=0}^{p-1}\sum_{j=0}^m\frac{a_{m,j}}{m!}n^{j+\frac{p-1}{2}}=\sum_{j=0}^m\frac{a_{m,j}}{m!}\sum_{n=0}^{p-1}n^{j+\frac{p-1}{2}}.
\]
Take a generator $g$. Then
\[
	\sum_{n=0}^{p-1}n^{j+\frac{p-1}{2}}=\sum_{n=0}^{p-1}(gn)^{j+\frac{p-1}{2}}=g^{j+\frac{p-1}{2}}\sum_{n=0}^{p-1}n^{j+\frac{p-1}{2}}
\]
so $\displaystyle\sum_{n=0}^{p-1}n^{j+\frac{p-1}{2}}=0$ because $0<j+\frac{p-1}{2}<p-1$. Thus $\displaystyle\sum_{n=m}^{p-1}\binom{n}{m}\left(\frac{n}{p}\right)=0$. But if $m=\frac{p-1}{2}$, then
\[
	\sum_{n=m}^{p-1}\binom{n}{m}\left(\frac{n}{p}\right)=\sum_{j=0}^m\frac{a_{m,j}}{m!}\sum_{n=0}^{p-1}n^{j+\frac{p-1}{2}}=\frac{a_{m,m}}{m!}(p-1)=-\frac{1}{\left(\frac{p-1}{2}\right)!}
\]
so
\[
	G(1+Y)\equiv-\frac{1}{\left(\frac{p-1}{2}\right)!}Y^{\frac{p-1}{2}}\pmod{Y^{\frac{p+1}{2}}}.
\]
Now we expand $H(1+Y)$. Note that $(1+Y)^{-k/2}-(1+Y)^{k/2}\equiv-kY\pmod{Y^2}$ so
\[
	H(1+Y)\equiv(-1)(-2)\cdots\left(-\frac{p-1}{2}\right)Y^{p-1}{2}\equiv\frac{(p-1)!}{\left(\frac{p-1}{2}\right)!}Y^{\frac{p-1}{2}}\pmod{Y^{\frac{p+1}{2}}}.
\]
But $G(1+Y)\equiv\eps H(1+Y)\pmod\Phi_p(1+Y)$ and $\Phi_p(1+Y)=Y^{p-1}$, so $G(1+Y)\equiv\eps H(1+Y)\pmod Y^{\frac{p+1}{2}}$. It follows that
\[
	-1\equiv\eps(p-1)!\pmod Y
\]
so by Wilson's Theorem, $\eps=1$.

Revert to $\mathbb{C}$. We have $G(\zeta)=H(\zeta)$. Check that $\zeta^{-k/2}-\zeta^{k/2}=-2i\sin\frac{2\pi(k/2)}{p}$ (where $k/2$ is taken mod $p$). This is a positive multiple of $i$ when $k$ is odd and a negative multiple of $i$ when $k$ is even. Thus every odd-even consecutive pair is a positive real number times $i\cdot(-i)=1$. It follows that $H(\zeta)$ is either along the positive real axis or positive imaginary axis. Since $g_p=G(\zeta)=H(\zeta)$ and $|g_p|=\sqrt{p}$, the result follows.