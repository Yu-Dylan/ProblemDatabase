No, in fact $\frac{2}{5}$ is not necessarily in $S$.

Assume $\frac{2}{5}$ is necessarily in $S$. Then there must be a finite sequence of operations $x+1$, $x-1$, and $\frac{1}{x(x-1)}$ we can make to go from $0$ to $\frac{2}{5}$. Consider the last time the $\frac{1}{x(x-1)}$ operation is used; let it go from $\frac{a}{b}$ (in reduced form) to $\frac{2}{5}+n$ for some integer $n$. Then \[\frac{b^2}{a(a-b)}=\frac{5n+2}{5}\] for some integers $a,b,n$ with $\gcd(a,b)=1$. But then $(b^2,a(a-b))=(5n+2,5)$ or $(-5n-2,-5)$ whence $b^2\equiv2$ or $3\pmod5$. This is a contradiction.