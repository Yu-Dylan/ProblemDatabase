Define the polynomials $A(x)=\displaystyle\sum_{i=1}^nx^{a_i}$ and $B(x)=\displaystyle\sum_{i=1}^nx^{b_i}$. Write \[A(x)^2=\displaystyle\sum_{i=1}^nx^{a_i}\displaystyle\sum_{j=1}^nx^{a_j}=\displaystyle\sum_{1\leq i,j\leq n}x^{a_i+a_j}.\] This sum adds $x^{a_i+a_j}$ twice for each $1\leq i<j\leq n$ and $x^{2a_i}$ once for each $1\leq i\leq n$. Since $\displaystyle\sum_{i=1}^nx^{2a_i}=A(x^2)$, we have that $\frac{A(x)^2-A(x^2)}{2}$ is the sum of $x^{a_i+a_j}$ over $1\leq i<j\leq n$. By the problem statement, this is the sum of $x^{b_i+b_j}$ over $1\leq i<j\leq n$. But by symmetry, this sum is $\frac{B(x)^2-B(x^2)}{2}$, so \[A(x)^2-A(x^2)=B(x)^2-B(x^2)\] as a polynomial identity. Rearrange this to \[\left(A(x)-B(x)\right)\left(A(x)+B(x)\right)=A(x^2)-B(x^2).\] Now, let $A(x)-B(x)=P(x)(x-1)^k$ for a non-negative integer $k$ and polynomial $P$ where $x-1$ does not divide $P$. This is possible because $A(x)-B(x)$ is not the zero polynomial (else the two sets are the same), so by the factor theorem we can repeatedly factor out $x-1$ from $A(x)-B(x)$ until we cannot any more. Then by the factor theorem, $P(1)\neq0$. First note that $A(1)=B(1)=n$, so $k\geq1$. Now, we have \[P(x)(x-1)^k\left(A(x)+B(x)\right)=P(x^2)(x^2-1)^k=P(x^2)(x+1)^k(x-1)^k.\] Since this is a polynomial identity, we can divide by $(x-1)^k$ to get \[P(x)\left(A(x)+B(x)\right)=P(x^2)(x+1)^k.\] Plug in $x=1$ to get \[P(1)\left(A(1)+B(1)\right)=P(1)2^k.\] Since $P(1)\neq0$ and $A(1)=B(1)=n$, we deduce that $n=2^{k-1}$, a power of $2$.