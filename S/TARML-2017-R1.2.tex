Let $Q\left(x\right)=x^3+Tx+1$. Note that $P=Q$ and $P=-Q$ both work. Now, assume that $P$ is a proper non-constant factor of $Q$. Note that any proper factor of $Q$ must have degree at most $2$. First, assume that a linear factor $x-r$ divides $Q$. Then $r$ is a root of $Q$ by the Factor Theorem. By the Rational Root Theorem, $r$ is either $1$ or $-1$. But $r^3+Tr+1=0$, so then $T$ is either $2$ or $-2$. We now assume that $T$ is neither $2$ nor $-2$ and only care about them if we get passed back one of these values. Then $Q$ has no linear factors. But then if $Q$ has any quadratic factors, then the quotient must be a linear term, contradiction. Thus, $Q$ also has no quadratic factors.

Thus, if $T\neq2,-2$, then $P$ can only be either $Q$ or $-Q$ for $2$ solutions. Since we have that $T=33$, the answer is $\boxed{2}$.

Alternative Solution: Let $Q\left(x\right)=x^3+Tx+1$ and $R\left(x\right)=x^3+Tx^2+1$. Note that $x^3Q\left(\frac{1}{x}\right)=R\left(x\right)$. Assume that $\left|T\right|>2$. I claim that $Q$ is irreducible over the integers. Note that it suffices to prove that $R$ is irreducible over the integers. But Perron's Criterion gives that $R$ is irreducible, so this is true. Since $T=33$, $\left|T\right|>2$, so $Q$ is irreducible over the integers. Thus, the only $P$ that work are $Q$ and $-Q$, so the answer is $\boxed{2}$.