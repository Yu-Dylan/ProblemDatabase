\begin{center}
	\begin{asy}
		size(15cm);
		pair A=(0,sqrt(80)), B=(-1,0), C=(1,0), I=(0,2/sqrt(5)), I_A=(0,-sqrt(5)/2), I_B=(9,sqrt(80)), I_C=(-9,sqrt(80));
		real r=2/sqrt(5), r_A=sqrt(5)/2, r_B=4*sqrt(5);
		pair X=I_A+r_A*dir(I--I_A), Y=I_B-r_B*dir(I--I_B), Z=I_C-r_B*dir(I--I_C);
		draw(A--B--C--cycle,green+linewidth(1.1));
		draw(circle(I,r+2r_A),blue+linewidth(1.1));
		draw(circle(I,r),rgb(218,112,214)); draw(circle(I_A,r_A),rgb(218,112,214)); draw(circle(I_B,r_B),rgb(218,112,214)); draw(circle(I_C,r_B),rgb(218,112,214));
		draw((-17,0)--B); draw(C--(17,0)); draw(B-1/3*(A-B)--B); draw(A--A+8/9*(A-B)); draw(C-1/3*(A-C)--C); draw(A--A+8/9*(A-C));
		draw(I--I_B,dashed); draw(I--I_C,dashed); draw(I--X,dashed);
		dot(A); dot(B); dot(C); dot(I); dot(I_A); dot(I_B); dot(I_C); dot(X); dot(Y); dot(Z);
	\end{asy}
\end{center}

Let $I,I_B$ be the incenter and $B$-excenter, let $r,r_A,r_B$ be the inradius, $A$-exradius, $B$-exradius. Then by circle tangency we deduce that \[r+2r_A=II_B-r_B.\] Using the fact that $\frac{r_A}{r}=\frac{s}{s-a}=\frac{a+2b}{2b-a}$ and $\frac{r_B}{r}=\frac{s}{s-b}=\frac{a+2b}{a}$, we have that \[II_B=r+2r_A+r_B=r\left(1+\frac{2a+4b}{2b-a}+\frac{a+2b}{a}\right)=r\cdot\frac{2b\left(3a+2b\right)}{a\left(2b-a\right)}.\] But you can use your favorite computing methods to compute \[BI=\frac{a}{a+2b}\sqrt{b\left(a+2b\right)}\] from which it follows that \[BI_B=\frac{ab}{BI}=\sqrt{b\left(a+2b\right)}\] so \[II_B=\frac{2b}{a+2b}\sqrt{b\left(a+2b\right)}\] while \[r=\frac{K}{s}=\frac{a\sqrt{\left(a+2b\right)\left(2b-a\right)}}{2\left(a+2b\right)}\] so \[\frac{2b}{a+2b}\sqrt{b\left(a+2b\right)}=\frac{a\sqrt{\left(a+2b\right)\left(2b-a\right)}}{2\left(a+2b\right)}\cdot\frac{2b\left(3a+2b\right)}{a\left(2b-a\right)}.\] After mass cancellation, we are left with \[\sqrt{b}=\frac{3a+2b}{2\sqrt{2b-a}}\] which simplifies to $\left(9a-2b\right)\left(a+2b\right)=0$. So $9a=2b$ and thus $a=2,b=9$ to give a perimeter of $\boxed{020}$ as desired.