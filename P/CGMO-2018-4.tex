There are $n$ students with distinct names. To start, each student has $n-1$ envelopes with the other students' names. Everyone also has at least one greeting card with their name on it. Every day, one student encloses a greeting card in their possession (possibly one which was sent to them) in an envelope (which the student must have had at the start) such that the card and envelope do not say the same name. The student then sends the envelope with the card inside to the student whose name is on the envelope (the mail arrives on the same day).

Suppose that at some point in time, no student can send a card in the above manner.
\begin{enumerate}[label=(\roman*)]
	\item Prove that everyone still has at least one card.
	\item If there are students $P_1,\ldots,P_k$ such that $P_i$ never sent a card to $P_{i+1}$ for $i=1,\ldots,k$ (indices taken modulo $k$), prove that $P_1,\ldots,P_k$ had the same number of greeting cards to start.
\end{enumerate}