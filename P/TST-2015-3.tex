A physicist encounters $2015$ atoms called usamons. Each usamon either has one electron or zero electrons, and the physicist can't tell the difference. The physicist's only tool is a diode. The physicist may connect the diode from any usamon $A$ to any other usamon $B$. (This connection is directed.) When she does so, if usamon $A$ has an electron and usamon $B$ does not, then the electron jumps from $A$ to $B$. In any other case, nothing happens. In addition, the physicist cannot tell whether an electron jumps during any given step.  The physicist's goal is to isolate two usamons that she is sure are currently in the same state. Is there any series of diode usage that makes this possible?